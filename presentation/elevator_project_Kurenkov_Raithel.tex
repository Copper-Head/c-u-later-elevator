\documentclass{beamer}
\usepackage{pgfpages}

\setbeameroption{show notes}
\setbeameroption{show notes on second screen}

\usetheme{metropolis}           % Use metropolis theme (https://github.com/matze/mtheme)
\usepackage{listings}
\usepackage{color}

\definecolor{codegreen}{rgb}{0,0.6,0}
\definecolor{codegray}{rgb}{0.5,0.5,0.5}
\definecolor{codepurple}{rgb}{0.58,0,0.82}
\definecolor{backcolour}{rgb}{0.95,0.95,0.92}
\definecolor{codeblue}{rgb}{0.102, 0.102, 0.255}

\lstdefinestyle{mystyle}{
    backgroundcolor=\color{backcolour},
    commentstyle=\color{codegreen},
    identifierstyle=\color{codeblue},
    keywordstyle=\color{magenta},
    numberstyle=\tiny\color{codegray},
    stringstyle=\color{codepurple},
    basicstyle=\footnotesize,
    breakatwhitespace=false,
    breaklines=true,
    captionpos=b,
    keepspaces=true,
    numbers=left,
    numbersep=5pt,
    showspaces=false,
    showstringspaces=false,
    showtabs=true,
    tabsize=4,
    literate={\ \ }{{\ \ }}1,
}

\lstset{style=mystyle}

\title{Advanced Problem Solving Techniques\\Final Project}
\date{\today}
\author{Ilia Kurenkov\\Lisa Raithel}
% (kurenkov@uni-potsdam.de)
% (lraithel@uni-potsdam.de)
\institute{University of Potsdam\\MSc. Cognitive Systems}

\begin{document}
  \maketitle

  \section{Implementation}

  \begin{frame}{Implementation -- Intuition}

    \begin{itemize}
    	\item build globally optimal plan
    	\item elevators move according to plan
    \end{itemize}

  \end{frame}
  \note[itemize]{
    \item not really what we were supposed to do
    \item not possible to add new floors while running
    \item ...
  }
%%%%%%%%%%%%%%%%%%%%%%%%%%%%%%%%%%%%%%%%%%%%%%%%%%%%%%%%%%%%%%%

  \begin{frame}{Implementation -- Idea I}

    \begin{itemize}
    	\item naive solution: graph navigation problem
    	\item way too slow
    	\item couldn't handle more than five requests and one elevator $\rightarrow$ too complex
    \end{itemize}

  \end{frame}

    \note[itemize]{
      \item treated problem as graph navigation problem
      \item discovered this approach is way too slow
    }

%%%%%%%%%%%%%%%%%%%%%%%%%%%%%%%%%%%%%%%%%%%%%%%%%%%%%%%%%%%%%%%

\begin{frame}{Implementation -- Idea II}

    \begin{itemize}
    	\item derive solution incrementally in the style of a finite state machine
    	\item fast, but failed for some instances
    	\item difficult to understand and debug

    \end{itemize}

\end{frame}

\note[itemize]{
    \item coolest idea
    \item treat elevators and their ``ways'' as a finite state machine
    \item elevator goes from state to state, like ``serving'', ``moving to target floor'', ``being done''
    \item fastest (?) solution so far, but failures
    \item with more complex instances we got into problems, hard to debug what caused the failures
  }


%%%%%%%%%%%%%%%%%%%%%%%%%%%%%%%%%%%%%%%%%%%%%%%%%%%%%%%%%%%%%%%


  \begin{frame}{Implementation -- Helpful Clues}

    \begin{itemize}
		\item care about floors, not only requests
		\begin{itemize}
			\item floors that need to be served
			\item floors with nothing to do
		\end{itemize}
		\item some requests have to be served by specific elevators $\rightarrow$ less choice
		\item divide requests equally amongst elevators
		\item compute travel distance without recursively constructing paths through a graph
  	\item specifying of plan in the first shot iteration
    \end{itemize}

  \end{frame}

  \note[itemize]{
    \item some clues we used: the actual floor position, not only requests
    \item distinguish about target floors and non-target floors
    \item deliver requests have to be served by a specific elevator $\rightarrow$ elevators have to go at least in this direction and will likely pass other floors on their way
    \item so save time, that is, steps, divide requests equally
    \item calculate travel distances to keep track of time/steps
    \item in the end: specified more and more of the ``travel plan'' before the elevators actually moved
    \item \textbf{venn diagram of an example 3 elevators and how requests would be divided amongst them?}
  }


%%%%%%%%%%%%%%%%%%%%%%%%%%%%%%%%%%%%%%%%%%%%%%%%%%%%%%%%%%%%%%%

    \begin{frame}{Implementation -- Idea III}

    \begin{itemize}
    	\item identify floors that need to be served
    	\item add deliver requests to corresponding elevator
    	\item distribute remaining request equally
        \begin{itemize}
          \item no elevator should be allowed to idle while another is still working
          \item \texttt{move} as small a distance as possible
        \end{itemize}
    \end{itemize}



    \begin{itemize}
    	\item behave differently when there's only one elevator
    \end{itemize}

  \end{frame}

  \note[itemize]{
    \item wanted to have target floors and ignore the rest (only operation in remaining floors: \texttt{move})
    \item deliver requests had assigned elevator, so we added them to the elevators ``target list''
    \item wanted to assign the remaining \texttt{calls} equally
    \item elevators should move as little as possible to save time
    \item somewhat different behavior for single elevators since they have to serve all floors
  }

%%%%%%%%%%%%%%%%%%%%%%%%%%%%%%%%%%%%%%%%%%%%%%%%%%%%%%%%%%%%%%%

% \begin{frame}[fragile]{Implementation -- Idea III (\#program base.)}

% \begin{lstlisting}[language=Prolog]
% % each elevator is free to choose as targets
% % any subset of the remaining free calls.
% { target(E, F) : free_call(F) } :-
%         agent(elevator(E)),
%         not only_one_elevator.

% % Note that every "free call" must be served.
% :- free_call(F), not target(_, F).

% \end{lstlisting}
% \end{frame}

% \note[itemize]{
%   \item first snippet:
%   \begin{itemize}
%     \item distribute the remaining targets amongst the elevators
%     \item (if there is more than one elevator)
%   \end{itemize}
%   \item second snippet: make sure that every call has an elevator taking care of it
% }

%%%%%%%%%%%%%%%%%%%%%%%%%%%%%%%%%%%%%%%%%%%%%%%%%%%%%%%%%%%%%%%

\begin{frame}[fragile]{Implementation -- \#program base.}

  \begin{itemize}
    \item check if there is a request right on the floor where the elevator starts
    \item predicate that assigns ``coordinates'' from elevator to target
    \item get the distance to target furthest along some direction
  \end{itemize}

\begin{lstlisting}[language=Prolog]
% Create coordinates from elevator to target
target_coord(E, -1, F, | DIST |) :-
        init(at(elevator(E), FE)),
        target(E, F),
        DIST = FE - F,
        DIST > 0.

\end{lstlisting}
\end{frame}

\note[itemize]{
  \item make sure that requests on the starting floor are served
  \item designed predicate that assigns the ``way'' from current elevator position to target position (direction in which to go, distance to travel)
  \item calculated distance to the targets with the greatest distance $\rightarrow$ after traveling this distance, the elevator doesn't have to go further (done with all requests)

}

%%%%%%%%%%%%%%%%%%%%%%%%%%%%%%%%%%%%%%%%%%%%%%%%%%%%%%%%%%%%%%%

\begin{frame}[fragile]{Implementation -- \#program base.}

  \begin{itemize}
    \item if elevator has targets above and below, the starting direction is direction of nearest target
    \item count \texttt{move} and \texttt{serve} operations
    \item combine number of \texttt{move} and \texttt{serve} operations to get the number of steps
  \end{itemize}

\begin{lstlisting}[language=Prolog]
% Combine serves and moves to get number of steps,
% which we then try to minimize.
n_steps(E, STEPS) :-
        STEPS = ST + D,
        total_n_serves(E, ST),
        total_n_moves(E, D),
        agent(elevator(E)).

\end{lstlisting}


\end{frame}

\note[itemize]{
  \item determine the direction in which an elevator moves first $\rightarrow$ depends on distance to first target
  \item count how many \texttt{moves} in total occur on each side (direction) of the elevator's starting position
  \item if elevator needs to change directions (targets on both sides of starting position), the distance that is traveled twice should be the smaller distance
  \item same goes for the number of \texttt{serves}
  \item combine those numbers to get the number of need steps to complete a single elevator's mission
  \item this is also an aspect we want to minimize later
}

\begin{frame}[fragile]{Implementation -- \#program base.}
  Elevator Instructions:
  \begin{itemize}
    \item fix the serving times: determine \textit{where} and \textit{when} the elevator has to serve
    \item determine when elevators have to switch direction
    \item determine the current direction for each time step
  \end{itemize}

\begin{lstlisting}[language=Prolog]
% For bidirectional elevators we need to know
% when they switch direction
switch_point(E, STEP) :-
        bidirectional(E),
        initial_direction(E, DIR),
        furthest_along(E, DIR, DIST),
        n_serves(E, DIR, STOPS),
        STEP = STOPS + DIST.
\end{lstlisting}

\end{frame}


\note[itemize]{
  \item general elevator instructions $\rightarrow$ elevators act according to their predefined plan
  \item make sure to serve requests at the starting positions if there are any
  \item determine serving ``where's'' and ``when's'' $\rightarrow$ elevators have fixed schedule, calculations based on travel distances and previously made stops and moves
  \item set the switching directions time step $\rightarrow$ when elevator has reached target that was the furthest above or below
  \item make sure the elevator knows in each time step n which direction he has to move next
}

%%%%%%%%%%%%%%%%%%%%%%%%%%%%%%%%%%%%%%%%%%%%%%%%%%%%%%%%%%%%%%%

\begin{frame}[fragile]{Implementation -- \#program step(t).}
  \begin{itemize}
    \item derive \texttt{do} and \texttt{holds} predicates based on \texttt{serve\_at} plans
    \item elevator moves if it does not serve and there are remaining targets
    \item carry requests along if they are not served yet
    \item update elevator positions after every \texttt{move}
  \end{itemize}

\begin{lstlisting}[language=Prolog]
% serve if the plan says so
do(elevator(E), serve, t) :-
        serve_at(E, F, t),
        holds(at(elevator(E), F), t - 1).

% move if there is something left to do
do(elevator(E), move(DIR), t) :-
        current_direction(E, DIR, t),
        not serve_at(E, _, t).

\end{lstlisting}

\end{frame}

\note[itemize]{
  \item entire incremental part is derived from the previously determined plan
  \item elevators only serve if the plan says ``serve at this time step''
  \item if there is something left to do (\texttt{current\_direction()}), the elevator has to move (no idling)
  \item if a request is not served, then it is carried along (if there was no serving on the request floor)
  \item update the position after each time step
}
%%%%%%%%%%%%%%%%%%%%%%%%%%%%%%%%%%%%%%%%%%%%%%%%%%%%%%%%%%%%%%%


\section{Optimization}

\begin{frame}[fragile]{Optimization}
  \begin{itemize}
    \item get number of steps the slowest elevator has to make
    \item get combined travel distances of all elevators
    \item minimize those, priority on minimizing the needed steps
  \end{itemize}

\begin{lstlisting}[language=Prolog]
% minimizing steps is more important
% than traveled distance
#minimize{ 1@5,S : elevator_step(S) }.
#minimize{ 1@2,T : travel_distance(T) }.
\end{lstlisting}
\end{frame}


\note[itemize]{
  \item get the maximum number of steps the slowest elevator has to make
  \item slowest elevator = elevator with most requests/longest distance
  \item sum up travel distances of all elevators
  \item first minimize the steps, then the overall travel distance
}

%%%%%%%%%%%%%%%%%%%%%%%%%%%%%%%%%%%%%%%%%%%%%%%%%%%%%%%%%%%%%%%

%%%%%%%%%%%%%%%%%%%%%%%%%%%%%%%%%%%%%%%%%%%%%%%%%%%%%%%%%%%%%%%

  \section{Results}

  \begin{frame}{Results}
  \begin{itemize}
  	\item no timeouts
  	\item slowest instance: test case 58 with 11927 ms
  	\item overall time: 149028 ms
  \end{itemize}

  \end{frame}

\note[itemize]{
  \item ...
}

%%%%%%%%%%%%%%%%%%%%%%%%%%%%%%%%%%%%%%%%%%%%%%%%%%%%%%%%%%%%%%%

\end{document}
